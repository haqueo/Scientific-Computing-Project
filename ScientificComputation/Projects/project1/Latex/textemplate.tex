\documentclass[paper=a4, fontsize=12pt]{scrartcl} % A4 paper and 11pt font size

\usepackage[T1]{fontenc} % use 8-bit encoding that has 256 glyphs
\usepackage{fourier}     % use the Adobe Utopia font for the document
                         % (comment this line to return to the LaTeX default)
\usepackage[english]{babel} % English language/hyphenation
\usepackage{amsmath,amsfonts,amsthm} % math packages
\usepackage{subeqnarray}

\usepackage{lipsum} % used for inserting dummy 'Lorem ipsum' text into the template
\usepackage{bold-extra}

\usepackage{listings}
\usepackage[utf8]{inputenc}

% default fixed font does not support bold face
\DeclareFixedFont{\ttb}{T1}{txtt}{bx}{n}{11} % for bold
\DeclareFixedFont{\ttm}{T1}{txtt}{m}{n}{11}  % for normal

% custom colors
\usepackage{color}
\definecolor{deepblue}{rgb}{0,0,0.5}
\definecolor{deepred}{rgb}{0.6,0,0}
\definecolor{deepgreen}{rgb}{0,0.5,0}
\definecolor{lightblue}{rgb}{0.95,0.95,1}
\definecolor{lightgrey}{rgb}{0.6,0.6,0.6}
\usepackage{listings}

% use graphics packages
\usepackage{graphicx}
\usepackage{float}
\usepackage{tikz}
\usetikzlibrary{matrix}
\usetikzlibrary{calc}
\usetikzlibrary{patterns,fadings}

% python style for highlighting
\newcommand\pythonstyle{\lstset{
language=Python,
backgroundcolor=\color{lightblue},
basicstyle=\ttm,
    % add keywords here
keywordstyle=\ttb\color{deepblue},
emph={while,for,if,elif,else,def,as,shape,conj,dot,copy,flatten,eye,zeros,ones,hstack,vstack,real,imag,conjugate,sin,cos,exp,append,insert,index,__main__}, % custom highlighting
%emphstyle=\ttb\color{deepred},     % custom highlighting style
emphstyle=\ttb\color{deepblue},     % custom highlighting style
stringstyle=\color{deepgreen},
commentstyle=\color{lightgrey},
frame=tb,                         % any extra options here
numbers=left,
showstringspaces=false            %
}}

% python environment
\lstnewenvironment{python}[1][]
{
\pythonstyle
\lstset{#1}
}
{}

% python for external files
\newcommand\pythonexternal[2][]{{
\pythonstyle
\lstinputlisting[#1]{#2}}}

% python for inline
\newcommand\pythoninline[1]{{\pythonstyle\lstinline!#1!}}


\usepackage{sectsty}        % allows customizing section commands
\allsectionsfont{\centering \normalfont\scshape}      % make all sections centered
                                                      % the default font and small caps

\usepackage{fancyhdr}        % custom headers and footers
\pagestyle{fancyplain}       % makes all pages in the document conform to
                             % the custom headers and footers
\fancyhead{}                 % no page header - if you want one, create it in
                             % the same way as the footers below
\fancyfoot[L]{}              % empty left footer
\fancyfoot[C]{}              % empty center footer
\fancyfoot[R]{\thepage}      % page numbering for right footer
\renewcommand{\headrulewidth}{0pt}     % remove header underlines
\renewcommand{\footrulewidth}{0pt}     % remove footer underlines
\setlength{\headheight}{13.6pt}        % customize the height of the header

\numberwithin{equation}{section}       % number equations within sections
                                       % (i.e. 1.1, 1.2, 2.1, 2.2 instead of 1, 2, 3, 4)
\numberwithin{figure}{section}         % number figures within sections
                                       % (i.e. 1.1, 1.2, 2.1, 2.2 instead of 1, 2, 3, 4)
\numberwithin{table}{section}          % number tables within sections
                                       % (i.e. 1.1, 1.2, 2.1, 2.2 instead of 1, 2, 3, 4)

\setlength\parindent{0pt}         % removes all indentation from paragraphs
                                  % comment this line for an assignment with lots of text

%--------------------------
%	TITLE SECTION
%--------------------------

\newcommand{\horrule}[1]{\rule{\linewidth}{#1}} % create horizontal rule command
                                                % with 1 argument of height

\title{
\normalfont \normalsize
\textsc{Imperial College London, Department of Mathematics} \\ [25pt]
\horrule{0.5pt} \\[0.4cm]                      % thin top horizontal rule
\huge Scientific Computing (M3SC) Project 1 \\           % the assignment title
\horrule{2pt} \\[0.5cm]                        % thick bottom horizontal rule
}

\author{Omar Haque}
\date{\normalsize\today}

\begin{document}
%\ttfamily
%\fontseries{b}\selectfont

\maketitle

\section{Main Solution}

The code in \textit{solution.py} contains the main program which carries out the process outlined by the question, i.e modelling the process of the cars moving across the city of Rome using the rules described. I have added scripts \textit{to be added} and \textit{to be added} to help answer the related questions at the end of the project.
\newline

Below are the imports used by the main program.


\begin{python}
# Imports
import Dijkstra as dijk
import misc
import numpy as np
import csv

# This import is needed for the last question
from solution_accident_occurs import max_index_tracker_no30


# ------------------------------------------------
# -----------    FUNCTIONS USED     --------------
# ------------------------------------------------

def next_node(path):
    """ Returns the next index (after the node itself) in the path.
        If the path contains only one node, returns the node itself.
    """
    if len(path) == 1:
        return path[0]
    else:
        return path[1]


def update_weight_matrix(epsilon, c, original_weight_matrix, noNodes=58):
    """
    This function updates the weight matrix according to step 5 of the
    Project. Note the added fix - the weight matrix is not changed if
    the original entry was 0.



    :param epsilon: given in question
    :param c: the vector containing number of cars at each node
    :param original_weight_matrix: the weight matrix given by RomeEdges
    :param noNodes: number of nodes in the system
    :return: the updated weight matrix
    """
    new_weight_matrix = np.zeros((noNodes, noNodes))
    for i in range(noNodes):
        for j in range(noNodes):
            if original_weight_matrix[i, j] != float(0):
                new_weight_matrix[i, j] = original_weight_matrix[i, j] + \
                                          (epsilon * (float(c[i]) +
                                                      float(c[j]))) / float(2)
    return new_weight_matrix


def extract_data():
    """
    This function opens the RomeVertices and RomeEdges files, and creates
    global variables RomeX, RomeY, RomeA, RomeB and RomeV. These are variables
    used to create the original weight matrix.

    """
    global RomeX, RomeY, RomeA, RomeB, RomeV
    RomeX = np.empty(0, dtype=float)
    RomeY = np.empty(0, dtype=float)
    with open('./data/RomeVertices', 'r') as file:
        AAA = csv.reader(file)
        for row in AAA:
            RomeX = np.concatenate((RomeX, [float(row[1])]))
            RomeY = np.concatenate((RomeY, [float(row[2])]))
    file.close()
    RomeA = np.empty(0, dtype=int)
    RomeB = np.empty(0, dtype=int)
    RomeV = np.empty(0, dtype=float)
    with open('./data/RomeEdges2', 'r') as file:
        AAA = csv.reader(file)
        for row in AAA:
            RomeA = np.concatenate((RomeA, [int(row[0])]))
            RomeB = np.concatenate((RomeB, [int(row[1])]))
            RomeV = np.concatenate((RomeV, [float(row[2])]))
    file.close()

\end{python}

My solution for moving the cars across the graph are as follows.

For each iteration (minute)

\end{document}
